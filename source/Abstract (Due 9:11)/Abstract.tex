\documentclass[letter,10pt,english]{article}

\usepackage[a4paper]{geometry}
\newgeometry{left=1in,right=1in,top=1in,bottom=1in}

\usepackage{fancyhdr}
\renewcommand{\headrulewidth}{0pt}
\fancyhf{}
\fancyhead[L]{\normalsize{ME163: Abstract}}
\fancyhead[C]{}
\fancyhead[R]{\normalsize{Advisor: Stan Wei}}
\fancyfoot[L]{}
\fancyfoot[C]{}
\fancyfoot[R]{}

\begin{document}

\title{Design of a Carbon Fiber 3D Printer for High Strength Applications}
\author{Peter Ascoli \& Jackie Song}
\date{}

\maketitle

\thispagestyle{fancy}

%Problem Definition

The current array of Fused Deposition Molding (FDM) 3D printers produce relatively weak parts. They print using weak thermoplastics, most commonly polylactic acid (PLA) or acrylonitrile butadiene styrene (ABS), and deposit short fibers in planar layers parallel to the build platform. This additive manufacturing method yields printed parts that fail quickly under loading due to layers being sheared or pulled apart; failure is particularly evident in parts with thin bosses or shells normal to the build plane, which offer very little surface area for layers to bond. Some parts can be oriented in space in the printer software to minimize this issue, but cannot be completely eliminated. These faults in FDM printers limit much of 3D printing to rapid prototyping applications (where strength is not necessary) even though there is high demand for printers that can produce stronger parts that can be used in load-bearing applications.\\

%Senior Project Ambitions

We intend to create stronger 3D printed parts by creating a printer capable of depositing fibrous materials using curved layer FDM. Using fibrous materials, such as carbon fiber, will allow for parts with increased mechanical strength properties if the fiber can be optimally oriented along a load path. Depositing material in curved layers instead of using the standard Cartesian method will permit fiber alignment and can also be utilized to create a smoother surface finish. Experimental comparisons between curved layer FDM and standard FDM parts will quantify the increase in strength and provided insight towards optimizing fiber orientation.\\

%General Plan of Attack

To realize the goal of a carbon fiber curved layer 3D printer we will first learn to use to the \emph{mini robot}. This robot arm provides \emph{6 degrees of freedom} and with a mounted extruder will allow us to create standard Cartesian FDM prints, as well as curved prints with control over fiber orientation. Before mounting an extruder to the arm, we first plan to familiarize ourselves with the arm and control software by developing tool paths within its operational envelope. After gaining control of the arm, an extruder can be mounted to the arm to create standard thermoplastic prints FDM prints. From this point on, we plan to experiment with orientation of the extruder to create curved layer thermoplastic prints, while simultaneously investigating the best fibrous material to use as printer material. The extruder will then be tweaked and or redesigned to realize full 3D printing capabilities with fibrous materials. Finally, with the curved layer fibrous 3D printing achieved, we plan to experimentally compare the mechanical strength of curved layer fibrous FDM prints to thermoplastic FDM prints; the Instron machine in the Material Sciences Laboratory can be used to load the parts and strain gauges can be attached to the parts to monitor specific local stresses.

%Research Conclusions



\end{document}
