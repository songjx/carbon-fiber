\documentclass[letter,10pt]{article}

\usepackage[a4paper]{geometry}
\newgeometry{left=1in,right=1in,top=1in,bottom=1in}

\usepackage{fancyhdr}
\pagestyle{plain}
\renewcommand{\headrulewidth}{0pt}
\fancyhead{}
\fancyhead[L]{ME163: Abstract}
\fancyhead[C]{}
\fancyhead[R]{Advisor: Stan Wei}
\fancyfoot{}
\fancyfoot[L]{}
\fancyfoot[C]{}
\fancyfoot[R]{}

\begin{document}

\title{Design of a Fiber 3D Printer for Higher Strength Applications}
\author{Peter Ascoli \& Jackie Song}
\date{}
\maketitle

%Introduction Paragraph

The current array of Fused Deposition Molding (FDM) 3D printers produce relatively weak parts. They print using weak thermoplastics, most commonly polylactic acid (PLA) or Acrylonitrile butadiene styrene (ABS), and deposit short fibers in planar layers parallel to the build platform. This additive manufacturing method yields printed parts that fail quickly under loading due to layers being sheared or pulled apart; failure is particularly evident in parts with thin bosses or shells normal to the build plane, which offer very little surface area for layers to bond. Some parts can be oriented in space in the printer software to minimize this issue, but we want to eliminate these issues entirely. Therefore, we intend to create a printer capable of depositing fibrous materials using curved layers in order to strengthen parts and expand 3D printing manufacturing beyond rapid prototyping applications. Furthermore, we intend to experimentally test these improved 3D printed parts and quantify the increase in strength relative to standard FDM parts.

%Body Paragraph(s)


\end{document}