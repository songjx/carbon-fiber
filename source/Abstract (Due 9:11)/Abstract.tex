\documentclass[letter,10pt]{article}

\usepackage[a4paper]{geometry}
\newgeometry{left=1in,right=1in,top=1in,bottom=1in}

\usepackage{fancyhdr}
\pagestyle{plain}
\renewcommand{\headrulewidth}{0pt}
\fancyhead{}
\fancyhead[L]{ME163: Abstract}
\fancyhead[C]{}
\fancyhead[R]{Advisor: Stan Wei}
\fancyfoot{}
\fancyfoot[L]{}
\fancyfoot[C]{}
\fancyfoot[R]{}

\begin{document}

\title{Design of a Fiber 3D Printer for Higher Strength Applications}
\author{Peter Ascoli \& Jackie Song}
\date{}
\maketitle

%Introduction Paragraph

The current array of Fused Deposition Molding (FDM) 3D printers produce relatively weak parts. They print using weak thermoplastics, most commonly polylactic acid (PLA) or acrylonitrile butadiene styrene (ABS), and deposit short fibers in planar layers parallel to the build platform. This additive manufacturing method yields printed parts that fail quickly under loading due to layers being sheared or pulled apart; failure is particularly evident in parts with thin bosses or shells normal to the build plane, which offer very little surface area for layers to bond. Some parts can be oriented in space in the printer software to minimize this issue, but we want to eliminate these issues entirely. Therefore, we intend to create a printer capable of depositing fibrous materials using curved layers in order to strengthen parts and expand 3D printing manufacturing beyond rapid prototyping applications. Curved layers allow fibers to be aligned with load paths and improve the surface finish of printed parts, while fibrous materials promise improved mechanical strength. Furthermore, we intend to experimentally test these improved 3D printed parts and quantify the increase in strength relative to standard FDM parts.\\

%General Plan of Attack

To realize these goals we will first learn to use to the \emph{mini robot}. This robot arm provides \emph{6 degrees of freedom} and with a mounted extruder will allow us to create standard Cartesian FDM prints, as well as curved prints with control of the fiber orientation. Before mounting an extruder to the arm, we first plan to familiarize ourselves with the arm and control software by developing tool paths within its operational envelope. After gaining control of the arm, an extruder can be mounted to the arm to create standard thermoplastic prints FDM prints. From this point on, we plan to experiment with orientation of the extruder to create curved layer thermoplastic prints. The extruder will then be tweaked and or redesigned to realize printing capabilities with fibrous materials. Finally, with the curved layer fibrous 3D printing achieved, we plan to experimentally compare the mechanical strength of curved layer fibrous FDM prints to thermoplastic FDM prints; the Instron can be used to load the parts and strain gauges can be attached to monitor local stress.

%Research Conclusions

\end{document}