\documentclass[letter,10pt,english]{article}

%test comment

\usepackage[a4paper]{geometry}
\newgeometry{left=1in,right=1in,top=1in,bottom=1in}

\usepackage{fancyhdr}
\renewcommand{\headrulewidth}{0pt}
\fancyhf{}
\fancyhead[L]{\normalsize{ME163 Fall 2014: Abstract}}
\fancyhead[C]{}
\fancyhead[R]{\normalsize{Advisor: Stan Wei}}
\fancyfoot[L]{}
\fancyfoot[C]{}
\fancyfoot[R]{}

\begin{document}

\title{Design of a Carbon Fiber 3D Printer for High Strength Applications}
\author{Peter Ascoli \& Jackie Song}
\date{}

\maketitle

\thispagestyle{fancy}

%Problem Definition

3D printing technologies offer the ability to produce new parts quickly, allowing faster and cheaper design iterations. However, commercially available Fused Deposition Modeling (FDM) 3D printers currently produce relatively weak parts due to the materials and print method used. They print using weak thermoplastics, most commonly polylactic acid (PLA) or acrylonitrile butadiene styrene (ABS), and deposit the material only in planar layers parallel to the build platform. Because of poor layer adhesion, this method yields parts that fail quickly under loading due to layers being sheared or pulled apart; this type of failure is particularly likely in parts with thin bosses or shells normal to the build plane, which offer very little surface area for layers to bond. Some parts may be rotated in the printer software to maximize layer contact area, but the planar layer limitation remains. These faults in FDM printers limit much of 3D printing to rapid prototyping applications (where strength is not necessary) despite high demand for printers that can produce stronger parts for use in load-bearing applications.\\

%Senior Project Ambitions

We intend to create stronger 3D printed parts by building a printer capable of depositing a continuous strand of fiber-reinforced polymer in curved layers. The material will mirror existing, widely used fiber-reinforced polymer materials, such as carbon fiber and fiberglass composites. Using a serial robot arm instead of a typical cartesian coordinate robot will allow the printer to lay material in any direction. Freedom from the planar layer limitation, combined with precise control over toolpaths, will allow the fibers to be oriented optimally for the desired mechanical properties of the part. These properties will be user-determined based on expected loading, surface finish, or other requirements. Experimental comparisons between curved layer FDM and standard FDM parts will quantify the increase in strength and provide insight towards optimizing fiber orientation.\\

%General Plan of Attack

To realize the goal of a fiber-reinforced composite curved layer 3D printer, we will first learn to use to the \emph{mini robot}. This robot arm provides the necessary \emph{6 degrees of freedom} for curved layer printing. We first plan to familiarize ourselves with the arm and control software by developing tool paths within its operational envelope. Then, we can mount an extruder to the arm and use an existing toolchain to create standard flat-layer thermoplastic FDM prints. From this point on, we plan to experiment with orientation of the extruder to create curved layer thermoplastic prints, while simultaneously investigating the best fiber composite to use as printing material. The extruder will then be tweaked and or redesigned to realize full 3D printing capabilities with fibrous materials. Finally, with the curved layer fibrous 3D printing achieved, we plan to experimentally compare the mechanical strength of curved layer fibrous FDM prints to thermoplastic FDM prints; the Instron machine in the Material Sciences Laboratory can be used to load the parts and strain gauges can be attached to the parts to monitor specific local stresses.

%Research Conclusions



\end{document}
