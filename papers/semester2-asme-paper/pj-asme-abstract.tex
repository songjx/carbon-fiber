%%% ABSTRACT %%%

Current desktop 3D printers use FDM (Fused Deposition Modeling) to build parts out of flat layers of extruded thermoplastics. The printed parts have poor mechanical properties because of the low strength of thermoplastics and because the flat layer geometry limits inter-layer adhesion in thin areas. A curved-layer CFRP (Carbon Fiber Reinforced Polymer) 3D printer was developed to solve those two issues. With curved layers, the carbon fiber may be oriented to best suit the applied loading on any given part, and the layers may be designed for greater inter-layer adhesion. An FDM-compatible ABS-matrix CFRP filament was developed by dipping a carbon fiber tow in an ABS-Acetone solution and was shown to have promising mechanical properties, comparable to aluminum. A custom FDM extruder was designed and fabricated for mounting to an available FANUC LR Mate 200iC industrial robot arm, which provides the six degrees of freedom needed to print curved layers. Control electronics, in the form of the FANUC robot system and open-source Megatronics 3D printer microcontroller board, were implemented and programmed to generate a toolpath for a sample specimen and operate the extruder hardware during printing. Finally, a composite-specific finite element analysis predicted the strength of the printed CFRP sample part to be twice that of ABS with stiffness on the same order as and aluminum part of the same geometry.