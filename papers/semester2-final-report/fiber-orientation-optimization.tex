\section{Finite Element Analysis}

\indent

Finite element analysis was used to predict the load-bearing capacity and failure behavior of the sample bridge specimen under the compressive load.\\

\emph{ANSYS} finite element software was used to determine fiber orientation within printed parts. The \emph{Composite PrepPost} package, which is specifically designed to assess the strength of composite structures, will be used. The package utilizes shell elements that contain matrix and fiber material properties. The software also contains multiple laminate failure criteria and can perform optimization loops to determine the strongest configuration of fiber angles. For instance, a tube can discretized into a number of shell elements, assigned a number of layers, given isotropic matrix material properties and orthotropic fiber material properties, and iterated through different fiber angles to calculate stress and deformation gradients across the entire part (at all angles or at only the strongest angle). Therefore, once the specific printed part geometry for the curved layer carbon fiber is determined, this software will be used to find optimal fiber orientation(s).\footnote{Dr. Wootton has advised that many computation analyses usually overshoot experimental strength results. However, experimental and theoretical results will typically agree in trends and the computational model will sufficiently locate the weakest area(s) of the geometry.}\\

\subsection{ACP Results}

%% Deformation

\begin{figure}[htp]
\centering
\includegraphics[width=1\textwidth]{./figures/fea/fea-acp-tot-def}
\caption{Total displacement FEA result.}
\label{fig:fea-acp-tot-def}
\end{figure}

\begin{figure}[htp]
\centering
\includegraphics[width=1\textwidth]{./figures/fea/fea-acp-x-def}
\caption{X direction displacement FEA result.}
\label{fig:fea-acp-x-def}
\end{figure}

\begin{figure}[htp]
\centering
\includegraphics[width=1\textwidth]{./figures/fea/fea-acp-y-def}
\caption{Y direction displacement FEA result.}
\label{fig:fea-acp-y-def}
\end{figure}

\begin{figure}[htp]
\centering
\includegraphics[width=1\textwidth]{./figures/fea/fea-acp-z-def}
\caption{Mesh overview of the composite model in \textit{ACP} with a solid-body display state applied.}
\label{fig:fea-acp-z-def}
\end{figure}

%%% Puck Failure Results

\begin{figure}[htp]
\centering
\includegraphics[width=1\textwidth]{./figures/fea/fea-acp-pfailure-notext}
\caption{Puck Failure contour plot.}
\label{fig:fea-acp-pfailure-notext}
\end{figure}

\begin{figure}[htp]
\centering
\includegraphics[width=1\textwidth]{./figures/fea/fea-acp-pfailure-layer-closeup}
\caption{Closeup of Puck Failure contour plot with tabulated failing layer.}
\label{fig:fea-acp-pfailure-layer-closeup}
\end{figure}

\begin{figure}[htp]
\centering
\includegraphics[width=1\textwidth]{./figures/fea/fea-acp-pfailure-mode-closeup}
\caption{Closeup of Puck Failure contour plot with tabulated failure modes.}
\label{fig:fea-acp-pfailure-mode-closeup}
\end{figure}

\clearpage

\subsection{Comparison to Puck Failure Criterion}

\clearpage

