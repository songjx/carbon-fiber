\section{Conclusion}

\indent

A curved-layer CFRP 3D printer was developed to print parts with greater strength than current thermoplastic FDM printers. With curved layers, the carbon fiber may be oriented to best suit the applied loading on any given part, and the layers may be designed for greater inter-layer adhesion. An FDM-compatible ABS-matrix CFRP filament was developed by dipping a carbon fiber tow in an ABS-Acetone solution and was shown to have promising mechanical properties, comparable to aluminum. A custom FDM extruder was designed and fabricated for mounting to an available FANUC LR Mate 200iC industrial robot arm, which provides the six degrees of freedom needed to print curved layers. Control electronics, in the form of the FANUC robot system and open-source Megatronics 3D printer microcontroller board, were implemented and programmed to generate a toolpath for a sample specimen and operate the extruder hardware during printing. Finally, a composite-specific finite element analysis predicted the strength of the printed CFRP sample part to be twice that of ABS with stiffness on the same order as and aluminum part of the same geometry. With the framework for the printer completely liad out, it is our hope that the curved layer carbon fiber printer is picked up as a senior project next year where students print and mechanically test the sample and other specimen, implement minor adjustments to the setup to achieve maximum part strength based on experimental findings, and ultimately print the next generation of 3D printed parts.\\

%A curved-layer CFRP 3D printer is under development to print parts with greater strength than current FDM printers. With curved layers, the carbon fiber may be oriented to best suit the applied loading on any given part, and the layers may be designed for greater inter-layer adhesion. An FDM-compatible ABS-matrix CFRP filament was developed and shown to have promising mechanical properties, comparable to aluminum. A custom FDM extruder was designed and prototyped for mounting on an available FANUC industrial robot arm, which provides the six necessary degrees of freedom to print curved layers. Control electronics were assembled and will be programmed to control the custom extruder and take input signals from the FANUC robot controller. A composite-specific finite element analysis software package was acquired and will be used to optimize the print layer geometry. Future work includes manufacturing the custom extruder; refining the filament production and test methods to create a printable CFRP; programming the robot and extruder controller; generating optimized layer geometries; and printing and testing the CFRP material. \\

%Current desktop 3D printers use FDM (Fused Deposition Modeling) to build parts out of flat layers of extruded thermoplastics. The printed parts have poor mechanical properties because of the low strength of thermoplastics and because the flat layer geometry limits inter-layer adhesion in thin areas. A curved-layer CFRP (Carbon Fiber Reinforced Polymer) 3D printer was developed to solve those two issues. With curved layers, the carbon fiber may be oriented to best suit the applied loading on any given part, and the layers may be designed for greater inter-layer adhesion. An FDM-compatible ABS-matrix CFRP filament was developed by dipping a carbon fiber tow in an ABS-Acetone solution and was shown to have promising mechanical properties, comparable to aluminum. A custom FDM extruder was designed and fabricated for mounting to an available FANUC LR Mate 200iC industrial robot arm, which provides the six degrees of freedom needed to print curved layers. Control electronics, in the form of the FANUC robot system and open-source Megatronics 3D printer microcontroller board, were implemented and programmed to generate a toolpath for a sample specimen and operate the extruder hardware during printing. Finally, a composite-specific finite element analysis predicted the strength of the printed CFRP sample part to be twice that of ABS with stiffness on the same order as and aluminum part of the same geometry.
